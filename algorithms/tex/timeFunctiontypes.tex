\documentclass[twocolumn]{article}
\usepackage{algorithm}
\usepackage{algpseudocode}
\usepackage[a4paper, margin=1in]{geometry}
\usepackage{amsmath}
\usepackage{amssymb}
\usepackage{graphicx}

\setlength{\parskip}{1em}
\setlength{\parindent}{0em}

\title{Types of Time Function}
\author{Mohammed Rizin \\ Unemployed}
\date{March 20, 2025}

\begin{document}
\maketitle

\section{Introduction}
So far we have seen a lot of functions and we calculated its time complexity. But we have not seen the types of time functions. In this article, we will see the types of time functions. 
Now let's see the types of time functions.
Then How does this happen? Let's see. this is the main topic of this article.
This way we can classify the time functions into different types. Why do we need to classify the time functions? Because we can easily understand the time complexity of the function.
This happens as follows.

\end{document}