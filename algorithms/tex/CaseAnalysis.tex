\documentclass{article}
\usepackage{amsmath}
\usepackage{pgfmath}
\usepackage{amssymb}
\usepackage{amsthm}
\usepackage{tikz}
\usepackage[a4paper, margin=0.8in]{geometry}
\usepackage{algorithm}
\usepackage{algpseudocode}
\usepackage{graphicx}
\usepackage{subcaption}
\usetikzlibrary{calc}
\usepackage{float}
\usepackage{caption}

\setlength{\parskip}{1em}
\setlength{\parindent}{0em}

\title{Case Analysis}
\author{Mohammed Rizin \\ Unemployed}

\date{\today}

\newcommand{\drawArray}[2]{%
    \newcount\arrayLength
    \begin{figure}[H]
        \centering
        \begin{tikzpicture}
            \coordinate (s) at (0, 0);
            \def\found{0}
            \arrayLength=0
            \foreach \num [count=\i from 0] in {#1} {
                \node[rectangle, draw, minimum size=6mm] at (s) {\num};
                \ifnum\num=#2
                    \node at ($(s)-(0,0.8)$) {$\underbrace{\i}_{key}$};
                    \xdef\found{1}
                \else
                    \node at ($(s)-(0,0.5)$) {\i};
                \fi
                
                \coordinate (s) at ($(s) + (0.6, 0)$);
                \ifnum\found=0
                \global\advance\arrayLength by 1
                \fi
            }
            \ifnum\found=0
                \node at ($0.45*(s) + (0, -1.0)$) {Not Found!};
            \fi
        \end{tikzpicture}
        \caption{Searching for key = #2 after \the\arrayLength \text{ steps}}
    \end{figure}
}

\begin{document}
\maketitle

\section{Linear Search Algorithm}
\newcommand{\A}{8, 6, 12, 5, 9, 7, 4, 3, 16, 18}

Suppose we have an array:

\expandafter\drawArray\expandafter{\A}{3}

\begin{algorithm}
    \caption{Linear Search Algorithm}
    \label{linear_search}
    \begin{algorithmic}[1]
        \Procedure{LinearSearch}{$A, n, x$}
        \For {$i \gets 0$ to $n-1$}
        \If {$A[i] = x$}
        \State Return $i$
        \EndIf
        \EndFor
        \State Return $-1$
        \EndProcedure
    \end{algorithmic}
\end{algorithm}

\expandafter\drawArray\expandafter{\A}{0}

Best Case $ B(n) \Longrightarrow$ Searching key element is present at index 0.  Then it will take $O(1)$. \\
Worst Case $ W(n) \Longrightarrow$ Either the element is absent or its at last index. Then it will take $O(n)$. \\
Average Case  $\Longrightarrow \frac{\text{all possible cases time}}{\text{no. of cases}}$\\
\\
Most of the cases, finding the average case time is not possible. But most of the cases it would be equal to Worst case time.

To calculate :
\[
\begin{aligned}
    Avg Time &= \frac{\sum_{i = 0}^{n} \text{time taken to find $i^{th}$ index element}}{n}\\
    AvgTime &= \frac{1+2+3+\cdots+n}{n}\\
    AvgTime &= \frac{\frac{n{(n+1)}}{2}}{n}\\
    AvgTime &= \frac{(n+1)}{2} \simeq O(n) \cong W(n)
\end{aligned}
\]

\end{document}