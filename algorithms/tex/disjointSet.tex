\documentclass{article}
\usepackage{amsmath}
\usepackage{algorithm}
\usepackage[a4paper, margin=0.8in]{geometry}
\usepackage{algpseudocode}
\usepackage{graphicx}
\usepackage{amssymb}
\usepackage{pifont}
\usepackage{tikz}
\usetikzlibrary{calc}
\usepackage{float}
\usepackage{caption}
\usepackage{abstract}
\usepackage{makeidx}
\makeindex

\setlength{\parskip}{0.5cm}
\setlength{\parindent}{0pt}

\title{Disjoint Sets}
\author{Mohammed Rizin \\ Umemployed}
\date{\today}

\begin{document}
\maketitle
\printindex

\section{Disjoint sets \& operations} 
\index{Disjoint sets & operations}
This is similar to the concept from sets in Mathematics but not exactly same. Disjoint set is modified for its proper usage in creating algorithms. The famous algorithm based on disjoint set is criskell's algorithm.
Lets us find How we can do 

\section{Detecting a cycle} \index{Detecting a cycle}

\section{Graphical Representation} \index{Graphical Representation}

\section{Array Representation} \index{Array Representation}

\section{Union and collapsing Find} \index{Union and collapsing Find}

\end{document}


