\documentclass{article}
\usepackage{algorithm}
\usepackage{algpseudocode}
\usepackage[a4paper, margin=1in]{geometry}
\usepackage{amsmath}
\usepackage{amssymb}
\usepackage{graphicx}
\usepackage{float}

\setlength{\parskip}{2pt}
\setlength{\parindent}{0pt}

\title{Time Complexity}
\author{Mohammed Rizin \\ Unemployed}

\begin{document}
\maketitle

\section{Time Complexity}

\subsection{Simple For Loop}
\begin{algorithm}[H]
    \caption{Simple for loop}\label{simple_for}
    \begin{algorithmic}
        \For{$i \gets 0$ \textbf{to} $n-1$}
            \State $....STMT....$
        \EndFor
    \end{algorithmic}
\end{algorithm}
\noindent
Time complexity: $O(n)$.

\subsection{Simple Reverse Loop}
\begin{algorithm}[H]
    \caption{Simple reverse loop}\label{simple_for2}
    \begin{algorithmic}
        \For{$i \gets n$ \textbf{down to} $1$}
            \State $....STMT....$
        \EndFor
    \end{algorithmic}
\end{algorithm}
\noindent
Time complexity: $O(n)$.

\subsection{For Loop with Step}
\begin{algorithm}[H]
    \caption{Simple for loop with step}\label{simple_for3}
    \begin{algorithmic}
        \For{$i \gets 0$ \textbf{to} $n-1$ \textbf{step} $2$}
            \State $....STMT....$
        \EndFor
    \end{algorithmic}
\end{algorithm}
\noindent
Time complexity: $O(n)$ (as $n/2$ is asymptotically $O(n)$).

\subsection{Nested Loops}
\begin{algorithm}[H]
    \caption{Nested loops}\label{simple_for4}
    \begin{algorithmic}
        \For{$i \gets 0$ \textbf{to} $n-1$}
            \For{$j \gets 0$ \textbf{to} $i-1$}
                \State $....STMT....$
            \EndFor
        \EndFor
    \end{algorithmic}
\end{algorithm}
\noindent
Time complexity: $O(n^2)$.

\begin{table}[H]
    \centering
    \begin{tabular}{|c|c|c|c|c|c|}
        \hline
        $i$ & $j$ & STMT & Total STMT & Total Time & Time Complexity \\
        \hline
        0 & 0 & 1 & 1 & 1 & 1 \\
        1 & 0 & 1 & 2 & 3 & 3 \\
        1 & 1 & 1 & 3 & 6 & 6 \\
        2 & 0 & 1 & 4 & 10 & 10 \\
        2 & 1 & 1 & 5 & 15 & 15 \\
        2 & 2 & 1 & 6 & 21 & 21 \\
        \hline
    \end{tabular}
\end{table}

The total number of executions is $n(n+1)/2$, so the time complexity is $O(n^2)$.

\subsection{Summation Loop}
\begin{algorithm}[H]
    \caption{Summation loop}\label{simple_for5}
    \begin{algorithmic}
        \State $p \gets 0$
        \For{$i \gets 1$ \textbf{while} $p \leq n$}
            \State $p \gets p + i$
        \EndFor
    \end{algorithmic}
\end{algorithm}

\begin{table}[H]
    \centering
    \begin{tabular}{|c|c|c|c|c|c|c|}
        \hline
        $i$ & $p$ & STMT & Total STMT & Total Time & Time Complexity \\
        \hline
        1 & 1 & 1 & 1 & 1 & 1 \\
        2 & 3 & 1 & 2 & 3 & 3 \\
        3 & 6 & 1 & 3 & 6 & 6 \\
        4 & 10 & 1 & 4 & 10 & 10 \\
        5 & 15 & 1 & 5 & 15 & 15 \\
        6 & 21 & 1 & 6 & 21 & 21 \\
        \vdots & \vdots & \vdots & \vdots & \vdots & \vdots \\
        $k$ & $\frac{k(k+1)}{2}$ & 1 & $k$ & $\frac{k(k+1)}{2}$ & $\frac{k(k+1)}{2}$ \\
        \hline
    \end{tabular}
\end{table}

Assuming $p \leq n$:
\[
\begin{aligned}
    p &= \frac{k(k+1)}{2}, \\
    \frac{k(k+1)}{2} &> n, \\
    k^2 &> n, \\
    k &> \sqrt{n}.
\end{aligned}
\]
Time complexity: $O(\sqrt{n})$.

% \newpage
\subsection{Multiplication Loop}
\begin{algorithm}[H]
    \caption{Multiplication loop}\label{simple_for6}
    \begin{algorithmic}
        \For{$i \gets 1$ \textbf{while} $i \leq n$ \textbf{step} $2 \cdot i$ }
            \State$...Statement...$
        \EndFor
    \end{algorithmic}
\end{algorithm}

\begin{table}[H]
    \centering
    \begin{tabular}{|c|c|}
        \hline
        step & $i$\\
        \hline
        1 & 2\\
        2 & 4\\
        3 & 8\\
        4 & 16\\
        5 & 32\\
        \vdots & \vdots\\
        k & $2^k$\\
        \hline
    \end{tabular}
\end{table}
\[
\begin{aligned}
    \text{This stops when } i &> n: \\
    \text{i.e.  } 2^k &> n, \\
    \log_2{2^k} &> \log_2{n}\\
    k \cdot \log_2{2} &> \log_2{n}\\
    k &> \log_2{n}
\end{aligned}
\]
Time complexity: $O(\log_2{n})$.

However, If you notice log function, which gives both float and integer values, then should we take floor or ceil of the floating point value.

\begin{table}[H]
    \centering
    \begin{tabular}{|c|}
        \hline
        n = 8\\
        \hline
        1\\
        2\\
        4\\
        $8 < 8$ \text{fails !!}\\ 
        \hline
        So in total we could execute it for 3 iterations. $\log_2{8} = 3$
    \end{tabular}
\end{table}


\begin{table}[H]
    \centering
    \begin{tabular}{|c|}
        \hline
        n = 10\\
        \hline
        1\\
        2\\
        4\\
        8\\
        $16 < 10$ \text{fails !!}\\ 
        \hline
        So in total we could execute it for 4 iterations. $\log_2{10} = 3.32$\\
        $ceil(\log_2{10}) = 4$
    \end{tabular}
\end{table}

\subsection{Division Loop}
\begin{algorithm}
    \caption{For Loop with Division}\label{divloop}    
    \begin{algorithmic}
        \For{$i \gets n$  \textbf{while} $i \geq 1$  \textbf{step} $\frac{i}{2}$}
            \State Statement
        \EndFor
    \end{algorithmic}
\end{algorithm}

\begin{table}[H]
    \centering
    \begin{tabular}{|c|c|}
        \hline
        step & $i$\\
        \hline
        1 & $n$\\
        2 & $\frac{n}{2}$\\
        3 & $\frac{n}{2^2}$\\
        4 & $\frac{n}{2^3}$\\
        5 & $\frac{n}{2^4}$\\
        \vdots & \vdots\\
        k & $\frac{n}{2^{k-1}}$\\
        k+1 & $\frac{n}{2^k}$\\
        \hline
    \end{tabular}
\end{table}

\noindent The loop stops when:
\[
\begin{aligned}
    \frac{n}{2^{k-1}} &< 1, \\
    2^{k-1} &> n, \\
    k - 1 &> \log_2{n}, \\
    k &> \log_2{n} + 1.
\end{aligned}
\]
Thus, the time complexity is $O(\log_2{n})$.

\subsection{Square Loop}
\begin{algorithm}[H]
    \caption{Loop till square of i is less than n}
    \begin{algorithmic}
    \For{$i \gets 1$ \textbf{while} $i^2 < n$}
    \State Statement
    \EndFor
    \end{algorithmic}
\end{algorithm}

\begin{table}[H]
    \begin{tabular}{|c|c|}
        \hline
        i & $i^2$\\
        \hline
        0 & 0\\
        1 & 1\\
        2 & 4\\
        3 & 9\\
        \vdots & \vdots\\
        k & $k^2$\\
        \hline
    \end{tabular}
\end{table}

\noindent The loop stops when:
\[
\begin{aligned}
    k^2 &\geq n\\
    k &\geq \sqrt{n}\\
\end{aligned}
\]
Thus, the time complexity is $O(\sqrt{n})$.

\subsection{Independent loop}
\begin{algorithm}[H]
    \caption{Independent for loops}\label{simple_for}
    \begin{algorithmic}
        \For{$i \gets 0$ \textbf{to} $n-1$}
            \State $....STMT....$
        \EndFor
        \For{$j \gets 0$ \textbf{to} $n-1$}
            \State $....STMT....$
        \EndFor
    \end{algorithmic}
\end{algorithm}
\noindent
Since this is not nested. They are independent loops. They have $O(n)$ for both the loops adding to $T(n) = 2n$. \\
But \textbf{Time complexity:} $O(n)$.

\subsection{One loop dependent on another}
\begin{algorithm}[H]
    \caption{Independent for loops}\label{simple_for}
    \begin{algorithmic}
        \State $p \gets 0$
        \For{$i \gets 0$ \textbf{while} $i < n$ \textbf{step} $2 \cdot i$}
            \State $p++$ \Comment{$ p = \log_2(n)$}
        \EndFor
        \For{$j \gets 0$ \textbf{while} $j < p$ \textbf{step} $2 \cdot j$}
            \State $...STMT...$ \Comment{$\log_2{p}$}
        \EndFor
    \end{algorithmic}
\end{algorithm}
\noindent
Thus this algorithm takes $O(\log{\log{n}})$\\

\section{Analysis of If \& While loops}
\begin{algorithm}[H]
    \caption{If and While loops}\label{simple_for}
    \begin{algorithmic}
        \State $i \gets 0$ \Comment{1}
        \While{$i < n$} \Comment{$n + 1$}
            \State Statement \Comment{$n$}
            \State $i++$ \Comment{$n$}
        \EndWhile
    \end{algorithmic}
\end{algorithm}

$T(n) = 2n + 2$\\
\textbf{Time complexity:} $O(n)$

\subsection{While with two variables}
\begin{algorithm}[H]
    \caption{While with two variables}\label{twovar_while}
    \begin{algorithmic}
        \State $i \gets 1$
        \State $j \gets 1$
        \While{$j < n$}
            \State Statement
            \State $j \gets j+i$
            \State $i \gets i + 1$
        \EndWhile
    \end{algorithmic}
\end{algorithm}

Let us trace the values of $i$ and $j$:
\begin{table}[H]
    \centering
    \begin{tabular}{|c|c|}
        \hline
        $i$ & $j$\\
        \hline
        1 & 1\\
        2 & 2\\
        3 & $2+2$\\
        4 & $2+2+3$\\
        5 & $2+2+3+4$\\
        6 & $2+2+3+4+5$\\
        \vdots & \vdots\\
        k & $2+2+3+4+\cdots + k-1$\\
        k+1 & $2+2+3+4+\cdots + k \approx \frac{k \cdot {(k+1)} }{2}$\\
        \hline
    \end{tabular}
\end{table}

The loop stops when:
\[
\begin{aligned}
    \frac{k(k+1)}{2} &\geq n, \\
    k^2 &\gtrapprox n, \\
    k &\gtrapprox \sqrt{n}.
\end{aligned}
\]
Thus, the time complexity is $O(\sqrt{n})$.

The equivalent for loop is:
\begin{algorithm}[H]
    \caption{For loop with two variables}\label{twovar_for}
    \begin{algorithmic}
        \For{$i \gets 1$ \textbf{while} $j < n$}
            \State Statement
            \State $j \gets j+i$
        \EndFor
    \end{algorithmic}
\end{algorithm}

\subsection{Example : GCD of two numbers}
\begin{algorithm}[H]
    \caption{GCD of two numbers}\label{gcd}
    \begin{algorithmic}
        \While{$a \neq b$}
            \If{$a > b$}
                \State $a \gets a - b$
            \Else
                \State $b \gets b - a$
            \EndIf
        \EndWhile
    \end{algorithmic}
\end{algorithm}

Lets trace the values of $a = 10$ and $b = 15$:
\begin{table}[H]
    \centering
    \begin{tabular}{|c|c|c|}
        \hline
        $a$ & $b$ & $a \neq b$\\
        \hline
        10 & 15 & True\\
        5 & 15 & True\\
        5 & 10 & True\\
        5 & 5 & False\\
        \hline
    \end{tabular}
\end{table}
So it took 3 iterations to find the GCD of 10 and 15.\\
\textbf{Time complexity:} $O(\max(a,b))$
min Time complexity: $O(1)$ and max Time complexity: $O(n)$

\subsection{Example : Test Algorithm}
\begin{algorithm}[H]
    \caption{Test Algorithm}\label{test}
    \begin{algorithmic}
        \Procedure{Test}{$n$}
            % if n < less tha 5 print n else run a for loop from 0 to n-1 and print i
            \If{$n < 5$}
                \State \textbf{print} $n$
            \Else
                \For{$i \gets 0$ \textbf{to} $n-1$}
                    \State \textbf{print} $i$
                \EndFor
            \EndIf
        \EndProcedure
    \end{algorithmic}
\end{algorithm}
This algorithm has two parts:
\begin{enumerate}
    \item If $n < 5$, then it prints $n$.
    \item If $n \geq 5$, then it prints $0$ to $n-1$.
\end{enumerate}
Thus, the time complexity is $O(n)$ at worst. At best, it is $O(1)$.

\section{Conclusion}
\begin{algorithm}[H]
    \begin{algorithmic}
        \For {$i \gets 0$ \textbf{to} $n-1$}
        \State Statement \Comment{$\blacktriangleright O(n)$}
        \EndFor\\
        \For {$i \gets 0$ \textbf{to} $n-1$ \textbf{step} $i+2$}
        \State Statement \Comment{$\frac{n}{2} \blacktriangleright O(n)$}
        \EndFor\\
        \For {$i \gets n$ \textbf{to} $1$}
        \State Statement \Comment{$\blacktriangleright O(n)$}
        \EndFor\\
        \For {$i \gets 0$ \textbf{to} $n-1$ \textbf{step} $i \cdot 2$}
        \State Statement \Comment{$\blacktriangleright O(\log_2{n})$}
        \EndFor\\
        \For {$i \gets 0$ \textbf{to} $n-1$ \textbf{step} $i \cdot 3$}
        \State Statement \Comment{$\blacktriangleright O(\log_3{n})$}
        \EndFor\\
        \For {$i \gets n$ \textbf{to} $1$ \textbf{step} $\frac{i}{2}$}
        \State Statement \Comment{$\blacktriangleright O(\log_2{n})$}
        \EndFor\\
    \end{algorithmic}
\end{algorithm}

\end{document}
