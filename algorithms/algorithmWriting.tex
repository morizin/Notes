\documentclass[]{article}
\usepackage{algorithm}
\usepackage{algpseudocode}
\usepackage[a4paper, margin=1in]{geometry}
\usepackage{amsmath}

\title{How to Write and Analyze an Algorithm}
\author{Mohammed Rizin \\ Unemployed}


\begin{document}
\maketitle

\section{How to Write an algorithm}
\noindent
\begin{algorithm}
\caption{Swapping two numbers}\label{swap}
\begin{algorithmic}
\Procedure{Swap}{$a, b$}
\State$a \rightarrow temp$
\State$b := a$
\State$b \gets temp$
\EndProcedure{}
\end{algorithmic}
\end{algorithm}

We can write algorithm in any format. The intent behind writing an algorithm is to make it easy for the programmer to quickly understand it and make it into program.

\section{How to Analyze an algorithm?}
\subsection{Criteria for Analyzing an algorithm}
\begin{enumerate}
\item \textbf{Time Complexity :} \\
This is the most important criteria for analyzing an algorithm. It is the measure of the amount of time taken by an algorithm to run as a function of the length of the input. It is denoted by $T(n)$, where $n$ is the size of the input. This is not the actual time taken by the algorithm to run, but a Time Function.

\item \textbf{Space Complexity :}\\
Since the program going to be implemented and executed on a computer, We need to know how much space does the algorithm consumes. 

\item \textbf{Network data transfer:}\\
But thats not all, maybe we run the algorithm on a cloud server, then we How much data is being transferred over the network. Is there any unnecessary data or large data being transferred.

\item \textbf{Power Consumption:}\\
If we are running the algorithm on a mobile devices(such as laptops, smartphones, etc...), then we need to know how much power does the algorithm consumes.

\item \textbf{CPU Register Usage:}\\
How much CPU registers are being used by the algorithm.

\end{enumerate}

The criteria depends on the project you're working. Based on the project requirements, you can choose the criteria for analyzing the algorithm.

\subsection{Example of Analyzing an algorithm}
\noindent
\begin{algorithm}
\caption{Swapping two numbers}\label{analyze_swap}
\begin{algorithmic}
\Procedure{Swap}{$a, b$}
\State $temp \gets a$ \Comment{ 1 unit of time }
\State $a \gets b$ \Comment{ 1 unit of time }
\State $b \gets temp$ \Comment{ 1 unit of time }
\EndProcedure
\end{algorithmic}
\end{algorithm}

\subsubsection{Time Complexity}
Every single statement in the algorithm takes one unit of time.
$f(n) = 3$ \\
$O(n) = 1$ \\
\break
\noindent
For eg, \\
Although $x = 5*a + 3*b$ statement does have 4 operations, It only takes 1 unit of time. 
If you want to go in detail, you can consider the number of operations in the statement. You can also consider the machine code generated by the compiler.

\noindent
Suppose you are going to your friends house and you have a car, you can just take it and go. However, if you wanna go to Mars, you need to analyze every single details on how to contact NASA, how to get into the rocket, how to survive in space, etc... There is many aspects to consider. So in this case you need to analyze the algorithm in detail. It depends on the project requirements.

\subsubsection{Space Complexity}
The total number of variables used in the algorithm. \\
$S(n) = 3 words$  i.e. a, b, temp\\ 
In this case, the space complexity is 3.

$O(n) = 1$ \\ 
Even if it 3000 variables, the space complexity is 1 (i.e. $O(n) = 1$).
\\

\section{Frequency Count Method}
\subsection{Example}
\noindent
\begin{algorithm}
\caption{Sum of all elements in an array}\label{sum}
\begin{algorithmic}
\Procedure{Sum}{$A, n$}
\State$S \gets 0$
\For{$0$ to $n-1$}
    \State$S \gets S + A[i]$
\EndFor\EndProcedure\end{algorithmic}
\end{algorithm}

\end{document}

