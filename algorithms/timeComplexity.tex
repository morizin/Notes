\documentclass[twoside]{article}
\usepackage{algorithm}
\usepackage{algpseudocode}
\usepackage[a4paper, margin=1in]{geometry}
\usepackage{amsmath}
\usepackage{amssymb}
\usepackage{graphicx}
\usepackage{float}

\setlength{\parskip}{0pt} % Reduce paragraph spacing
\setlength{\parindent}{0pt} % Remove paragraph indentation

\title{Time Complexity}
\author{Mohammed Rizin \\ Unemployed}

\begin{document}
\maketitle

\section{Time Complexity}

\subsection{Simple For Loop}
\begin{algorithm}[H]
    \caption{Simple for loop}\label{simple_for}
    \begin{algorithmic}
        \For{$i \gets 0$ \textbf{to} $n-1$}
            \State $.... STMT ....$
        \EndFor
    \end{algorithmic}
\end{algorithm}
\noindent
Time complexity: $O(n)$.

\subsection{Simple Reverse Loop}
\begin{algorithm}[H]
    \caption{Simple reverse loop}\label{simple_for2}
    \begin{algorithmic}
        \For{$i \gets n$ \textbf{down to} $1$}
            \State $.... STMT ....$
        \EndFor
    \end{algorithmic}
\end{algorithm}
\noindent
Time complexity: $O(n)$.

\subsection{For Loop with Step}
\begin{algorithm}[H]
    \caption{Simple for loop with step}\label{simple_for3}
    \begin{algorithmic}
        \For{$i \gets 0$ \textbf{to} $n-1$ \textbf{step} $2$}
            \State $.... STMT ....$
        \EndFor
    \end{algorithmic}
\end{algorithm}
\noindent
Time complexity: $O(n)$ (as $n/2$ is asymptotically $O(n)$).

\subsection{Nested Loops}
\begin{algorithm}[H]
    \caption{Nested loops}\label{simple_for4}
    \begin{algorithmic}
        \For{$i \gets 0$ \textbf{to} $n-1$}
            \For{$j \gets 0$ \textbf{to} $i-1$}
                \State $.... STMT ....$
            \EndFor
        \EndFor
    \end{algorithmic}
\end{algorithm}
\noindent
Time complexity: $O(n^2)$.

\begin{table}[H]
    \centering
    \begin{tabular}{|c|c|c|c|c|c|}
        \hline
        $i$ & $j$ & STMT & Total STMT & Total Time & Time Complexity \\
        \hline
        0 & 0 & 1 & 1 & 1 & 1 \\
        1 & 0 & 1 & 2 & 3 & 3 \\
        1 & 1 & 1 & 3 & 6 & 6 \\
        2 & 0 & 1 & 4 & 10 & 10 \\
        2 & 1 & 1 & 5 & 15 & 15 \\
        2 & 2 & 1 & 6 & 21 & 21 \\
        \hline
    \end{tabular}
\end{table}

The total number of executions is $n(n+1)/2$, so the time complexity is $O(n^2)$.

\subsection{Summation Loop}
\begin{algorithm}[H]
    \caption{Summation loop}\label{simple_for5}
    \begin{algorithmic}
        \State $p \gets 0$
        \For{$i \gets 1$ \textbf{while} $p \leq n$}
            \State $p \gets p + i$
        \EndFor
    \end{algorithmic}
\end{algorithm}

\begin{table}[H]
    \centering
    \begin{tabular}{|c|c|c|c|c|c|c|}
        \hline
        $i$ & $p$ & STMT & Total STMT & Total Time & Time Complexity \\
        \hline
        1 & 1 & 1 & 1 & 1 & 1 \\
        2 & 3 & 1 & 2 & 3 & 3 \\
        3 & 6 & 1 & 3 & 6 & 6 \\
        4 & 10 & 1 & 4 & 10 & 10 \\
        5 & 15 & 1 & 5 & 15 & 15 \\
        6 & 21 & 1 & 6 & 21 & 21 \\
        \vdots & \vdots & \vdots & \vdots & \vdots & \vdots \\
        $k$ & $\frac{k(k+1)}{2}$ & 1 & $k$ & $\frac{k(k+1)}{2}$ & $\frac{k(k+1)}{2}$ \\
        \hline
    \end{tabular}
\end{table}

Assuming $p \leq n$:
\[
\begin{aligned}
    p &= \frac{k(k+1)}{2}, \\
    \frac{k(k+1)}{2} &> n, \\
    k^2 &> n, \\
    k &> \sqrt{n}.
\end{aligned}
\]
Time complexity: $O(\sqrt{n})$.

\end{document}
